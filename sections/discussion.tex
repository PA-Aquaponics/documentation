\documentclass[../main.tex]{subfiles}

\begin{document}
	\section{Diskussion und Ausblick}
	Die anfangs ausgestellten Anforderungen konnten wir erfüllen, doch je weiter wir vorankamen desto mehr Features wurden verlangt. So waren wir am Ende nicht in der Lage alle Features und Wünsche wie ein Authentifizierungssystem, eine \gls{mqtt}\cite{MQTT} Anbindung zu \gls{urbanblue}\cite{Urbanblue} und ein Alarmsystem zu implementieren.\\
	\par
	\noindent

	\subsection{Datenbank}
	Die in \gls{mysql} erstellte Datenbank hat wie in der Einleitung und in den Grundlagen erwähnt keine Beziehungen zwischen den Tabellen. Dies ist sehr suboptimal und hätte ebenfalls von uns überarbeitet werden können, jedoch haben wir uns dagegen entschieden, da weitere Änderungen vorgenommen werden müssen, hauptsächlich auf dem \gls{raspberrypi} welches ausserhalb unseres Fokus dieser Projektarbeit liegt. Doch muss noch dazu erwähnt werden, dass durch nicht Anpassen der Datenbank auf unserer Seite mehr Logik und Arbeit eingebaut werden musste. Durch genauere Analyse ist uns aufgefallen, dass die Logeinträge nicht mit den Sensoreinträgen verbunden sind, weil der vorherige Entwickler nicht wollte, dass bei Löschung der \gls{sensor}en auch die Logdaten verloren gingen. Dies wäre möglich sauberer in der Datenbank zu lösen, indem man ein zusätzliches Feld bei den Sensordaten hinzufügt um Sensoren als aktiviert oder deaktiviert zu markieren. Jedoch haben wir uns aus den schon erwähnten Gründen gegen die Änderung der Datenbankstruktur entschieden.
	
	\subsection{Erfahrungen und Meinungen zu den eingesetzten Tools}
	In den folgenden Kapiteln werden wir auf die eingesetzten Tools eingehen unsere Erfahrungen teilen und unsere Meinung dazu geben.
	
	\subsubsection{Github Actions}
	Wir hatten zwar schon Vorerfahrung mit \gls{githubactions}, aber simple würde man den Umgang mit \gls{githubactions} nicht bezeichnen. Trotzdem waren wir in der Lage \gls{githubactions} in unseren Workflow einzubauen. Es ist nicht nur nützlich für das Programm sondern auch für unsere Dokumentation die wir in TeX geschrieben haben. Das TeX Dokument wird automatisch beim \gls{push} oder \gls{merge} in den \gls{mainbranch} von \gls{githubactions} zu einem PDF kompiliert und auf \gls{github} in unserem \gls{repository} veröffentlicht. Unser Build and Publish Workflow für die komplette Applikation wird ebenfalls bei einer Veränderung im Main angestossen und diese Github Action geht noch einen Schritt weiter und erstellt ein \gls{docker} Image und pusht dieses auf unser \gls{dockerhub} Konto. Von dort aus kann es von überall heruntergeladen und in \gls{docker} ausgerollt werden.
	\begin{figure}[H]
		\centering
		\includegraphics[scale=0.5]{github_actions_all_workflows} 
		\caption{Github Actions}
		\label{fig:github_actions_all_workflows}
	\end{figure}
	
	\subsubsection{Docker und Docker-Compose}
	\gls{docker} alleine bringt schon eine Vielfalt an Funktionen aber mit \gls{dockercompose} war das Ausrollen von jedem neuen Release ein Kinderspiel. Da wir die Datenbank, \gls{phpmyadmin} und unser \gls{springboot} Server auf drei \gls{container} verteilt haben konnte der \gls{springboot} Server mit einem Befehl durch einen neuen Release ersetzt werden ohne dass wir die Daten verlieren und beim Fall unseres Hosts war das meist auch nur eine Dauer von maximal dreissig Sekunden bis der Server wieder funktionsfähig war.
	\begin{figure}[H]
		\centering
		\includegraphics[scale=0.4]{docker} 
		\caption{Docker mit laufenden Containern}
		\label{fig:docker}
	\end{figure}
	
	\subsubsection{Swagger}
	Die mit \gls{openapi} zu bedienende Swagger Software schreckte uns anfangs ein wenig ab. Doch schnell erkannten wir das Potential und die Vielfältigkeit der Software die uns behilflich sein konnte. Mithilfe von Online-Tools und SwaggerHub wird die Arbeit mit Swagger sehr erleichtert. Im Online-Tool gibt es eine schöne Vorschau und der \gls{yaml} Code der geschrieben wird, wird auch in Echtzeit überprüft und man wird auch auf Fehler hingewiesen. Mit dem verbundenem SwaggerHub kann man sich auch von \gls{api}s von anderen inspirieren lassen.
		\begin{figure}[H]
		\centering
		\includegraphics[scale=0.25]{swaggerhub} 
		\caption{SwaggerHub}
		\label{fig:swaggerhub}
	\end{figure}
	
	\subsubsection{Spring Boot}
	Die Arbeit mit Spring Boot nimmt einem einiges an \gls{boilerplate} ab. Zwar ist man in ein gewisses Framework gebunden, was durch die unzähligen, nötigen Annotationen ersichtlich wird, aber innerhalb dieses Frameworks sind die Konfigurationsmöglichkeiten immens. Die Tatsache, dass wir uns mittels OpenAPI-Generator einen Server Stub generieren konnten, war ein zweischneidiges Schwert. Einerseits war der Einstieg noch schneller, andererseits war es für uns, die noch wenige Berührungspunkte mit Spring Boot hatten, etwas schwierig, zu entziffern, welche Code-Stelle wofür zuständig war.
	
	\subsubsection{Angular}
	Angular und vorallem Typescript brachten für uns etwas Sicherheit und Altbekanntes in die Javascript Umgebung. Die Art und Weise, wie Angular-Projekte aufgebaut sind, kam uns auch gelegen, da sie den objektorientierten Designprinzipien, die wir im Laufe des Studiums angwendet haben, um einiges näher kommen, als wir es bisher aus Web bezogenen Projekten kannten.
	
	\noindent Zusätzlich zu Angular haben wir auch auf Angular Material, eine Style-Bibliothek für Angular, gesetzt. Funktionalitäten wie Dropdown-Menüs oder sogenannte Cards waren folglich schnell und einfach implementiert.
	
	\subsubsection{Sloppy.io}
	Dank Sloppy.io waren wir in der Lage, schnell und einfach verschiedene Docker-Container aufzusetzen. Zwar ist der Hosting Service von Sloppy.io im preislichen Mittelfeld, doch durch die einfache, zielgerichtete Handhabung (bspw. Container Networking) und die ebenfalls übersichtlichen Wartungsmöglichkeiten kamen wir zum Schluss, dass Sloppy.io für uns die Beste Option ist.
	
	\noindent Der einzige Nachteil aus unserer Sicht besteht darin, dass Sloppy.io lediglich bei der Business Option eine Möglichkeit zur Kollaboration mehrerer Teammitglieder bietet. Diese ist allerdings für kleine bis mittlere Projekte preislich nicht gerechtfertigt.
	
	\subsection{Bachelorarbeit Möglichkeiten}
	Weiterführende Möglichkeiten des Projektes bestehen. Unter anderem stehen Optionen offen, wie alle Logdateien als CSV zu exportieren um diese weiterzuverarbeiten.
	\\ \\
	Grössere weiterführende Projekte sind unter anderem ein Erkennungssystem für die Fische und die gesamte Umwelt im Tank. \\
	Folgende Punkte können zukünftig implementiert werden:
	\begin{itemize}
		\item Anzahl und Gewicht der Fische erkennen.
		\item Durch das Gewicht und die Anzahl die Fütterungsrate festlegen.
		\item Sauerstoffgehalt des Wassers erkennen und angezeigt werden. Damit man im Notfall jederzeit eingreifen kann.
		\item Gesundheit der Fische überwachen unter anderem:
		\begin{itemize}
			\item Körper und Flossenkonditionen des Fisches durch Kameras überwachen
			\item Schuppenprobleme, Deformierungen, Kiemenverfärbungen erkennen und frühzeitig warnen
		\end{itemize}
	\end{itemize}	
	Diese Punkte würden Bildbearbeitungsmethoden benötigen, welche diese Punkte erkennen können. Damit Massnahmen in den verschiedenen Fällen getroffen werden können. 
	\par 
	\begin{figure}[H]
		\centering
		\includegraphics{../images/Imageprocessing} 
		\caption{Bildbearbeitungsmethoden}
		\label{fig:Imageprocessing}
	\end{figure}
	
	\par \noindent
	Hiermit werden weitere Möglichkeiten für Überwachungen eröffnet wie z.B:
	
	\begin{itemize}
		\item Verhalten 
		\item Sterblichkeit
		\item Platzgebrauch
		\item Gruppierungen
		\item Isolationen
		\item Schwimmgeschwindigkeit
		\item Körpergleichgewicht
	\end{itemize}
	\noindent
	Das würde bereits Videobearbeitung erfordern, um das Verhalten auf gewisse Aktionen zu erkennen. 
	
\end{document}