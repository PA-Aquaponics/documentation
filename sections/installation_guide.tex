\documentclass[../main.tex]{subfiles}

\begin{document}
	\section{Installationsanleitung Allgemein}
	Durch den verwendeten Technologie-Stack lässt sich die Installation in kleinere Teilgebiete unterteilen. Die Voraussetzungen für eine erfolgreiche Installation sehen wie folgt aus:
	\begin{itemize}
		\item Java Version 11 oder höher
		\item Sofern der Spring-Boot-Server nicht als JAR-Datei vorhanden ist:
		\begin{itemize}
			\item Persönlicher GitHub-Account
			\item Gradle Version 7.3.+ oder Entwicklungsumgebung, die fähig ist, mit Gradle-Wrappern umzugehen.
		\end{itemize}
		\item Hosting-Angebot, das die Verwendung von Docker-Containern erlaubt
	\end{itemize}
	
	\subsection{Docker-Container}
	Der beabsichtigte Technologie-Stack sieht vor, dass zur erfolgreichen Inbetriebnahme zwei beziehungsweise drei Docker-Container verwendet werden.
	
	\subsubsection{Angular Frontend}
	\par Das Angular basierte Frontend wird durch den Spring Boot Server bereitgestellt. Bei der Erstellung des Servers wird dafür der Angular-Code kompiliert, optimiert und minimiert. Damit das Frontend auf die Korrekten Endpunkte zugreift, kann es nötig sein, die verwendeten Endpunkte anzupassen. Grundsätzlich reicht hierfür eine Anpassung vom \textit{basePath} in der Datei \texttt{environment.prod.ts} und eine weiter Anpassung vom \textit{API\_ENDPOINT} in der Datei \texttt{app-settings.ts}. Beide Dateien befinden sich im Unterordner \textit{paaq-client}.
	
	\subsubsection{Spring Boot Server}
	\par Durch die GitHub-Actions wird gewährleistet, dass auf DockerHub stets die aktuellste Version des Codes als Docker-Image verfügbar ist. Gemäss Grundeinstellungen der Actions ist dieses Image unter \textit{denadex/paaq} zu finden. 
	\par Die Anbindung an die Datenbank erfolgt über Parameter, die im Code gesetzt werden. Diese sind in der Datei \texttt{application.properties} im Unterordner \textit{paaq-server} aufgeführt.
	\par Zwischen dem Spring Boot Server und der Datenbank ist eine Verbindung essentiell. Das bedeutet, dass ein Port-Freigabe notwendig sein kann. Für genauere Angaben zum Setup einer Verbindung zwischen zwei Docker-Containern empfiehlt sich daher eine Konsultation des Hosting-Partners oder der Docker-Community.
	
	\subsubsection{MySQL Datenbank}
	\par Um über Docker-Container eine MySQL-Datenbank auszuführen, reicht es, das offizielle MySQL-Image zu verwenden. Dieses ist auf DockerHub unter \texttt{mysql} zu finden.
	\par Damit die anderen Container oder auch zusätzliche Dienste auf diese Datenbank zugreifen können, ist es notwendig, den Docker mit dem Parameter \texttt{MYSQL\_ROOT\_PASSWORD} zu starten. Als Passwort kann eine beliebige Zeichenfolge verwendet werden.
	
	\subsubsection{phpMyAdmin}
	\par Ein phpMyAdmin-Container ist für den erfolgreichen Betrieb des Front- und Backends nicht notwendig. Wenn eine phpMyAdmin jedoch gewünscht ist, um tiefgründigere Einblicke in die Datenbank zu erhalten, kann ein solcher Container zusätzlich verwendet werden.
	\par Auch für phpMyAdmin gibt es ein offizielles Docker-Image. Um einen phpMyAdmin-Container mit dem Datenbankcontainer zu verknüpfen müssen die Parameter \texttt{PMA\_HOST} und \texttt{MYSQL\_ROOT\_PASSWORD} gesetzt werden.
	\par Analog zum Spring Boot Server muss auch hier gewährleistet werden, dass eine Verbindung zwischen den Containern tatsächlich existieren kann.
	
	\subsubsection{Container Networking}
	Der Spring Boot Server ist über Port 8080 erreichbar. Damit man die Webseite also von ausserhalb erreichen kann, müssen HTTP/S Requests an den Container auf Port 8080 weitergeleitet werden.
	
	\subsubsection{Externe Hilfestellungen}
	\begin{itemize}
		\item Informationen zum MySQL Docker Image: \href{https://hub.docker.com/_/mysql}{Offizielles MySQL Docker Image}
		\item Informationen zum phpMyAdmin Docker Image: \href{https://hub.docker.com/r/phpmyadmin/phpmyadmin/}{Offizielles phpMyAdmin Docker Image}
		\item Informationen zu Container Networking: \href{https://docs.docker.com/config/containers/container-networking/}{Docker Dokumentation}
	\end{itemize}
	
	%-------------------------------------------------------------------------------------------
	
	\subsection{Eigene Anpassungen}
	Folgend eine Kurzübersichtig über die nötigen Anpassungen, falls eine eigene individualisierte Lösung in Betracht gezogen wird.
	\subsubsection{Eigene GitHub-Fork}
	Das Projekt ist Open-Source und kann somit öffentlich eingesehen werden. Für eigene Änderungen kann das GitHub-Repository mittels 'forking' geklont werden. In der geklonten Fork kann man wie bei GitHub üblich seine eigenen Änderungen einpflegen.
	\subsubsection{Eigenes Docker Image}
	\par Für ein eigenes Docker-Image ist ein Benutzerkonto bei DockerHub notwendig. Anschliessend muss ein neues Repository bei DockerHub erzeugt werden. Damit der Code der eigenen GitHub-Fork auf dieses Docker-Repository gelangt, muss in den GitHub-Actions die neue URL für das DockerHub-Repository hinterlegt werden. Anzupassen sind:
	\begin{itemize}
		\item In \texttt{.github/workflows/build\_and\_publish.yml}: In der Rubrik 'Build and Publish' kann das Tag zu einem anderen Namen als 'paaq' abgeändert werden.
		\item In den Einstellungen des GitHub-Repositorys müssen zwei Secrets hinterlegt sein:
		\begin{itemize}
			\item \texttt{DOCKER\_HUB\_USERNAME}: Der Benutzername des DockerHub-Benutzerkontos.
			\item \texttt{DOCKER\_HUB\_ACCESS\_TOKEN}: Ein Access-Token für das DockerHub-Benutzerkontos.
		\end{itemize}
	\end{itemize}
	
	%-------------------------------------------------------------------------------------------
	
	\section{Installationsanleitung mit Sloppy.io}
\end{document}