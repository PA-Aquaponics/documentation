\documentclass[../main.tex]{subfiles}

\begin{document}
	\section{Installationsanleitung Allgemein}
	Durch den verwendeten Technologie-Stack lässt sich die Installation in kleinere Teilgebiete unterteilen. Die Voraussetzungen für eine erfolgreiche Installation sehen wie folgt aus:
	\begin{itemize}
		\item Java Version 11 oder höher
		\item Sofern der \gls{springboot}-Server nicht als JAR-Datei vorhanden ist:
		\begin{itemize}
			\item Persönlicher GitHub-Account
			\item Gradle Version 7.3.+ oder Entwicklungsumgebung, die fähig ist, mit Gradle-Wrappern umzugehen.
		\end{itemize}
		\item Hosting-Angebot, das die Verwendung von Docker-Containern erlaubt
	\end{itemize}
	
	\subsection{Docker-Container}
	Der beabsichtigte Technologie-Stack sieht vor, dass zur erfolgreichen Inbetriebnahme zwei beziehungsweise drei Docker-Container verwendet werden.
	
	\subsubsection{Angular Frontend}
	\par \noindent Das Angular basierte Frontend wird durch den \gls{springboot} Server bereitgestellt. Bei der Erstellung des Servers wird dafür der \gls{angular} Code kompiliert, optimiert und minimiert. Damit das Frontend auf die Korrekten Endpunkte zugreift, kann es nötig sein, die verwendeten Endpunkte anzupassen. Grundsätzlich reicht hierfür eine Anpassung vom \textit{basePath} in der Datei \texttt{environment.prod.ts} und eine weiter Anpassung vom \textit{API\_ENDPOINT} in der Datei \texttt{app-settings.ts}. Beide Dateien befinden sich im Unterordner \textit{paaq-client}.
	
	\subsubsection{Spring Boot Server}
	\par \noindent Durch die \gls{githubactions} wird gewährleistet, dass auf \gls{dockerhub} stets die aktuellste Version des Codes als \gls{docker} Image verfügbar ist. Gemäss Grundeinstellungen der Actions ist dieses Image unter \textit{denadex/paaq} zu finden. 
	\par \noindent Die Anbindung an die Datenbank erfolgt über Parameter, die im Code gesetzt werden. Diese sind in der Datei \texttt{application.properties} im Unterordner \textit{paaq-server} aufgeführt.
	\par \noindent Zwischen dem \gls{springboot} Server und der Datenbank ist eine Verbindung essentiell. Das bedeutet, dass ein Port-Freigabe notwendig sein kann. Für genauere Angaben zum Setup einer Verbindung zwischen zwei \gls{docker} \gls{container} empfiehlt sich daher eine Konsultation des Hosting-Partners oder der \gls{docker} Community.
	
	\subsubsection{MySQL Datenbank}
	\par \noindent Um über \gls{docker} \gls{container}  eine \gls{mysql} Datenbank auszuführen, reicht es, das offizielle \gls{mysql} Image zu verwenden. Dieses ist auf \gls{dockerhub} unter \texttt{mysql} zu finden.
	\par \noindent Damit die anderen \gls{container} oder auch zusätzliche Dienste auf diese Datenbank zugreifen können, ist es notwendig, den \gls{docker} mit dem Parameter \texttt{MYSQL\_ROOT\_PASSWORD} zu starten. Als Passwort kann eine beliebige Zeichenfolge verwendet werden.
	\par \noindent Allfällige Daten aus früheren Datenbanken können mittels \gls{phpmyadmin} einfach in die neue \gls{mysql} Datenbank eingepflegt werden.
	
	\subsubsection{phpMyAdmin}
	\par \noindent Ein \gls{phpmyadmin} \gls{container} ist für den erfolgreichen Betrieb des Front- und Backends nicht notwendig. Wenn eine \gls{phpmyadmin} jedoch gewünscht ist, um tiefgründigere Einblicke in die Datenbank zu erhalten, kann ein solcher \gls{container} zusätzlich verwendet werden.
	\par \noindent Auch für \gls{phpmyadmin} gibt es ein offizielles \gls{docker} Image. Um einen \gls{phpmyadmin} \gls{container} mit dem Datenbank \gls{container} zu verknüpfen müssen die Parameter \texttt{PMA\_HOST} und \texttt{MYSQL\_ROOT\_PASSWORD} gesetzt werden.
	\par \noindent Analog zum \gls{springboot} Server muss auch hier gewährleistet werden, dass eine Verbindung zwischen den \gls{container} tatsächlich existieren kann.
	
	\subsubsection{Container Networking}
	Der \gls{springboot} Server ist über Port 8080 erreichbar. Damit man die Webseite also von ausserhalb erreichen kann, müssen HTTP/S Requests an den \gls{container} auf Port 8080 weitergeleitet werden.
	
	\subsubsection{Externe Hilfestellungen}
	\begin{itemize}
		\item Informationen zum \gls{mysql} \gls{docker} Image: \href{https://hub.docker.com/_/mysql}{Offizielles MySQL Docker Image}
		\item Informationen zum \gls{phpmyadmin} \gls{docker} Image: \href{https://hub.docker.com/r/phpmyadmin/phpmyadmin/}{Offizielles phpMyAdmin Docker Image}
		\item Informationen zu \gls{container} Networking: \href{https://docs.docker.com/config/containers/container-networking/}{Docker Dokumentation}
	\end{itemize}
	
	%-------------------------------------------------------------------------------------------
	
	\subsection{Eigene Anpassungen}
	Folgend eine Kurzübersichtig über die nötigen Anpassungen, falls eine eigene individualisierte Lösung in Betracht gezogen wird.
	\subsubsection{Eigene GitHub-Fork}
	Das Projekt ist Open-Source und kann somit öffentlich eingesehen werden. Für eigene Änderungen kann das \gls{github} \gls{repository} mittels 'forking' geklont werden. In der geklonten Fork kann man wie bei \gls{github} üblich seine eigenen Änderungen einpflegen.
	\subsubsection{Eigenes Docker Image}
	\par Für ein eigenes Docker-Image ist ein Benutzerkonto bei \gls{dockerhub} notwendig. Anschliessend muss ein neues \gls{repository} bei \gls{dockerhub} erzeugt werden. Damit der Code der eigenen \gls{github} Fork auf dieses \gls{docker} \gls{repository} gelangt, muss in den \gls{githubactions} die neue URL für das \gls{dockerhub} \gls{repository} hinterlegt werden. Anzupassen sind:
	\begin{itemize}
		\item In \texttt{.github/workflows/build\_and\_publish.yml}: In der Rubrik 'Build and Publish' kann das Tag zu einem anderen Namen als 'paaq' abgeändert werden.
		\item In den Einstellungen des \gls{github} \gls{repository} müssen zwei Secrets hinterlegt sein:
		\begin{itemize}
			\item \texttt{DOCKER\_HUB\_USERNAME}: Der Benutzername des \gls{dockerhub} Benutzerkontos.
			\item \texttt{DOCKER\_HUB\_ACCESS\_TOKEN}: Ein Access-Token für das \gls{dockerhub} Benutzerkontos.
		\end{itemize}
	\end{itemize}
	
	%-------------------------------------------------------------------------------------------
	
	\section{Installationsanleitung mit Sloppy.io}
	Da \gls{sloppyio} das Erstellen und Verbinden von Containern, sowie auch die Verwendung einer neuen URL erheblich erleichtert, eignet es sich gut, um Prototypen zu Demonstrationszwecken zu hosten. Nachfolgend wird das Setup mittels \gls{sloppyio} beschrieben.
	
	\subsection{Produktwahl}
	Für das Hosting der drei \gls{docker} \gls{container} reicht der 'Basic' Plan der von \gls{sloppyio} angebotenen Optionen aus. Sollte nach einiger Zeit der Speicher nicht mehr ausreichen oder mehr Rechenleistung nötig sein, können diese separat dazugekauft werden. Zusätzliche Features der Pläne 'Professional' und 'Business', wie beispielsweise 'External Logging' oder 'Custom Invoicing' sind für diesen Anwendungsfall unverhältnismässig. Ein Nachteil ist jedoch, dass Kollaboration mittels mehrerer Accounts erst mit dem teuersten Produkt möglich ist.
	
	\begin{figure}[H]
		\centering
		\includegraphics[width=0.8\textwidth]{../images/SloppyProductPlans} 
		\caption{Verschiedene Leistungsangebote von Sloppy.io}
		\label{fig:SloppyProductPlans}
	\end{figure}

	\subsection{Projekt erstellen}
	\gls{sloppyio} ermöglicht es Benutzern, ihre \gls{container} bzw. Services in Projekten zu gliedern. Es ist daher zu empfehlen, für die drei \gls{container} ein eigenes Projekt anzulegen. Das Projekt sowie auch die Services können mit einem Namen versehen werden, der die Übersicht erleichtert. Pro Service können beliebig viele \gls{container}-Apps angegliedert werden. Es ist jedoch empfohlen, nur stark voneinander abhängige \gls{container}-Apps im selben Service laufen zu lassen.
	
	\begin{figure}[H]
		\centering
		\includegraphics[width=0.8\textwidth]{../images/SloppyProjectStructure} 
		\caption{Strukturierung von Services und Apps in einem Sloppy-Projekt}
		\label{fig:SloppyProjectStructure}
	\end{figure}
	
	\subsection{MySQL-Datenbank}
	\par Für die Datenbank reicht es, per \gls{sloppyio} ein neues App für einen entsprechenden Service anzulegen. Anschliessend kann das \gls{docker}-Image direkt per Name und optionalem Version-Tag verwendet werden. Folgende Umgebungsvariablen sind notwendig:
	\begin{itemize}
		\item \texttt{MYSQL\_ROOT\_PASSWORD}: Ein geeignetes Passwort für den Root-Nutzer der Datenbank.
	\end{itemize}
	\par \noindent Für einen effizienten Betrieb der Datenbank sollten der \gls{container}-App mindestens 512MB Arbeitsspeicher zugeschrieben werden. Im Verlaufe der Entwicklung hat sich ein 1024MB grosser Arbeitsspeicher bewährt.
	
	\subsection{phpMyAdmin}
	\par Ein \gls{phpmyadmin}-\gls{container} kann analog zum \gls{mysql}-\gls{container} an einen entsprechenden Service gekettet und per \gls{sloppyio}-Dialog direkt von \gls{dockerhub} verwendet werden. Da \gls{phpmyadmin} von extern zugänglich sein soll, wird die App als öffentlich eingestellt. Als URL kann eine eigene verwendet werden oder man nutzt eine kostenfreie Subdomain von \gls{sloppyio}. 
	\par \noindent Folgende Umgebungsvariablen sind notwendig:
	\begin{itemize}
		\item \texttt{PMA\_HOST}: \gls{sloppyio} ermöglicht es dem Benutzer hier direkt den \gls{mysql}-\gls{container} anzugeben.
		\item \texttt{MYSQL\_ROOT\_PASSWORD}: Ein geeignetes Passwort für den Root-Nutzer der Datenbank.
	\end{itemize}
	
	\begin{figure}[H]
		\centering
		\includegraphics[width=0.8\textwidth]{../images/SloppyPHPPreferences} 
		\caption{Einstellungen für den phpMyAdmin Container.}
		\label{fig:SloppyPHPPreferences}
	\end{figure}
	
	\subsection{Spring Boot Server}
	\par Ein \gls{docker}-\gls{container} für den Spring Boot Server sollte ebenfalls in einem eigenen Service laufen. Analog zum \gls{phpmyadmin}- und dem \gls{mysql}-\gls{container} kann als \gls{docker}-Image dasjenige Image ausgewählt werden, dass auf \gls{dockerhub} erstellt wurde. Der \gls{springboot} Server \gls{container} muss ebenfalls öffentlich sein und sollte auf den Port 8080 hören.
	\par \noindent Umgebungsvariablen sind keine nötig, da diese bereits im \gls{container} Image gesetzt sind.
	
	\begin{figure}[H]
		\centering
		\includegraphics[width=0.95\textwidth]{../images/SloppySpringServer} 
		\caption{Der Spring Boot Server hört auf Port 8080.}
		\label{fig:SloppySpringServer}
	\end{figure}
	
	\subsection{Container Networking}
	\par Die einzelnen \gls{container} können über das interne Netzwerk von \gls{sloppyio} miteinander kommunizieren. Die dazu notwendige interne URL eines Containers kann eingesehen werden und dementsprechend im Code für den \gls{springboot} Server zur Anbindung der Datenbank hinterlegt werden.
	\begin{figure}[H]
		\centering
		\includegraphics[width=0.95\textwidth]{../images/SloppyConnectContainers} 
		\caption{Sloppy.io Container haben eine Sloppy.io interne Adresse.}
		\label{fig:SloppyConnectContainers}
	\end{figure}
	
\end{document}