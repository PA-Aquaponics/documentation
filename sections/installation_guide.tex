\documentclass[../main.tex]{subfiles}

\begin{document}
	\section{Installationsanleitung}
	Durch den verwendeten Technologie-Stack lässt sich die Installation in kleinere Teilgebiete unterteilen. Die Voraussetzungen für eine erfolgreiche Installation sehen wie folgt aus:
	\begin{itemize}
		\item Java Version 11 oder höher
		\item Sofern der Spring-Boot-Server nicht als JAR-Datei vorhanden ist:
		\subitem Persönlicher GitHub-Account
		\subitem Gradle Version 7.3.+ oder Entwicklungsumgebung, die fähig ist, mit Gradle-Wrappern umzugehen.
		\item Hosting-Angebot, das die Verwendung von Docker-Containern erlaubt
	\end{itemize}
	
	\subsection{Docker-Container}
	Der beabsichtigte Technologie-Stack sieht vor, dass zur erfolgreichen Inbetriebnahme zwei beziehungsweise drei Docker-Container verwendet werden. 
	\subsubsection{Spring Boot Server}
	\subsubsection{MySQL Datenbank}
	\subsubsection{phpMyAdmin}
	
	\subsection{Eigene Anpassungen}
	
\end{document}