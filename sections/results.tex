\documentclass[../main.tex]{subfiles}

\begin{document}
	\section{Resultate}
	
	\subsection{Frontend}
	
	Als Frontend haben wir 3 Menüpunkte ausgewählt, welche die Hauptfaktoren unserer Schnittstelle aufzeigen. \\
	Der erste Punkt ist die Übersicht. Hier werden die aktuellen Systeme mit Tanks und die Hauptsensorendaten aufgeführt. \\
	Unterhalb jedes Systems gibt es die Details Option. Über diesen gelangt man zu den detaillierten Infos jedes Tanks des oben genannten Systems in der unteren Abbildung \gls{angular} Frontend. 
	\begin{figure}[H]
		\centering
		\includegraphics[scale=0.4]{Frontend_Dashboard}
		\caption{Übersicht}
		\label{fig:Frontend_Dashboard}
	\end{figure}
	\begin{figure}[H]
		\centering
		\includegraphics[scale=0.4]{Frontend_Details}
		\caption{Logs}
		\label{fig:Frontend_Details}
	\end{figure}	
	\begin{figure}[H]
		\centering
		\includegraphics[scale=0.4]{Frontend_newSystem}
		\caption{Logs}
		\label{fig:Frontend_newSystem}
	\end{figure}
	\par
	\noindent	
	Das tabellarische Layout ist an das Design des \gls{sc1000} angelehnt, damit mit einer gewohnten Umgebung weitergearbeitet werden kann. 
	\\ \\
	Als letzter Menüpunkt wird noch auf die \gls{api} Dokumentation verwiesen, welche sich auf die Swagger UI bezieht. Dort sieht man alle Eigenschaften und andere Wissenswerte Details über die Schnittstelle für zukünftige Entwicklungen.\par 
	\begin{figure}[H]
		\centering
		\includegraphics[scale=0.4]{../images/API_Documentation}
		\caption{API Dokumentation}
		\label{fig:API_Documentation}
	\end{figure}
	\noindent
	Wie auf der Abbildung ersichtlich ist, werden verschiedene \gls{api} Aufrufe über das Aufrufen des Dashboards oder das Klicken auf den Editierbutton in den Sensorendaten gesteuert. 
	\\
	Somit wird das Arbeiten mit den Sensorendaten um ein Vielfaches erleichtert.
	
	%------------------------------------------------------------------
	
	\subsection{Backend}
	Unser Backend besteht aus einem Spring Boot Server und einer MySQL-Datenbank. Zudem besteht ist es Möglich, den Spring Boot Server in einem Docker Image zu verwenden.
	
	\subsubsection{Spring Boot Server}
	Spring Boot ermöglichte uns einen schnellen Start, indem es viele der \gls{boilerplate} lastigen Themen übernimmt und direkt zur Verfügung stellt. Die Anbindung der Datenbank, das Handhaben von Websockets und die Abhandlung von klassischen Fehlerquellen wie nicht gefundenen Resourcen seitens des Servers wurden somit direkt von Spring Boot gehandhabt. Somit konnten wir uns auf die Themen fokussieren, die unseren Server von anderen Servern unterscheiden wird, sprich wie er die Schnittstelle zur Verfügung stellen wird.
	
	\noindent Basierend auf der \gls{interfacefirst} Herangehensweise, konnten wir per OpenAPI-Generator einen Teil des Servers automatisch generieren. Datenmodelle, API und Controller liessen sich in einer ersten Form auf diese Weise rasch erzeugen. Die Verbindung der generierten Komponenten zur Datenbank wurde von uns manuell programmiert.
	
	\noindent Mittels Gradle werden durch Build-Scripts zuerst unser Frontend zu einem auslieferbaren Paket gebaut und anschliessend der Server-Code zusammen mit dem Angular-Paket in ein ausführbares JAR-File umgewandelt.
	
	\subsubsection{Datenbank}
	Als Datenbank verwenden wir eine MySQL-Datenbank, die mithilfe eines offiziellen MySQL Docker Image in einem Container läuft. Zusätzlich zum Datenbank-Container ist auch noch ein \gls{phpmyadmin} Container in Betrieb, der es dem Team der ZHAW Wädenswil ermöglicht, die Daten wie zuvor zu begutachten oder zu editieren.
	
	\noindent Als Datenbank-Schema haben wir das Schema der bisher vom Team der ZHAW Wädenswil gebrauchten Datenbank verwendet. Dies hauptsächlich deswegen, weil der Wunsch geäussert wurde, die alten Daten beizubehalten und auch den neuen Einarbeitungsaufwand klein zu halten.
	
	\subsubsection{Docker}
	Die zwei erwähnten Systeme laufen in drei \gls{docker} \gls{container}n da \gls{mysql} und \gls{phpmyadmin} in einzelnen \gls{container}n laufen. Alle drei \gls{container} sind in einem Containernetzwerk miteinander verbunden und können untereinander kommunizieren ohne dass der Verkehr nach aussen gelangt. Nur der \gls{springboot} \gls{container} und der \gls{phpmyadmin} \gls{container} ist von aussen erreichbar, da im Containernetzwerk ein \gls{portforwarding} für diese zwei \gls{container} eingerichtet wurden.
	
	\subsection{OpenAPI Generator und Swagger}
	Der \gls{interfacefirst}-Ansatz ermöglicht die Verwendung von einem OpenAPI-Generator. Mittlerweile gibt es diesen Generator in einer Open-Source Form und in der proprietären Version von Swagger, der ursprünglichen Entwicklerfirma. Wir haben uns für die Verwendung des Open-Source Generators entschieden.
	
	\noindent Mit dem Generator lassen sich eine Vielzahl von verschiedenen Fundamenten für Backends oder Frontends in unterschiedlichsten Programmiersprachen oder Frameworks erzeugen.
	
	\subsubsection{Swagger UI}
	Da Swagger ein prominenter Vorreiter der OpenAPI-Sphäre ist, unterstützt der generierte Spring Boot Server auch einige der verschiedenen Swagger-Funktionen. So kann unsere API ganz einfach eingesehen und direkt getestet werden.
	
\end{document}