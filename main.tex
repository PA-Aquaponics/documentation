\documentclass[a4paper]{article}

%------------------------------------------------------------
% PACKAGES
% Base Packages
\usepackage[ngerman]{babel}
\usepackage[utf8]{inputenc}
\usepackage[T1]{fontenc}
\usepackage{geometry}
\usepackage{subfiles}
\usepackage{xcolor}
\usepackage{graphicx}
\usepackage{pdfpages}
\usepackage{todonotes}
\usepackage{listings}
\usepackage[backend=bibtex, style=ieee, citestyle=ieee]{biblatex}
\usepackage{csquotes}

% Misc Packages
\usepackage{svg}
\usepackage{fancyhdr}
\usepackage{authblk}

% End Packages
%------------------------------------------------------------

%------------------------------------------------------------
% DEFINITIONS
% Color Definitions
\definecolor{codegreen}{rgb}{0,0.6,0}
\definecolor{codegray}{rgb}{0.5,0.5,0.5}
\definecolor{codepurple}{rgb}{0.58,0,0.82}
\definecolor{backcolour}{rgb}{0.921, 0.929, 0.937}

% Listings Definitions
\lstdefinestyle{mystyle}{
	backgroundcolor=\color{backcolour},   
	commentstyle=\color{codegreen},
	keywordstyle=\color{magenta},
	numberstyle=\tiny\color{codegray},
	stringstyle=\color{codepurple},
	basicstyle=\ttfamily\footnotesize,
	breakatwhitespace=false,         
	breaklines=true,                 
	captionpos=b,                    
	keepspaces=true,                 
	numbers=left,                    
	numbersep=5pt,                  
	showspaces=false,                
	showstringspaces=false,
	showtabs=false,                  
	tabsize=2
}
\lstset{style=mystyle}
%------------------------------------------------------------


%------------------------------------------------------------
% SETTINGS
% Bibliography settings
%\addbibresource{example.bib} % Add your own bibtex file here

% Geometry / Margin Settings
\geometry{
	paper=a4paper, % Change to letterpaper for US letter
	inner=2.5cm, % Inner margin
	outer=3.8cm, % Outer margin
	bindingoffset=.5cm, % Binding offset
	top=1.5cm, % Top margin
	bottom=1.5cm, % Bottom margin
	%showframe, % Uncomment to show how the type block is set on the page
}

% Paths
\graphicspath{{images/}}
\svgpath{{images/svg/}}
\pagestyle{fancy}
\fancyhf{}

% End Geometry / Margin Settings
%------------------------------------------------------------

%------------------------------------------------------------
% DOCUMENT SPECIFIC
% Author
\author{
	Deniz Akca
	\and
	Dennis Bannerman
	\and
	Mike Iten
}
\affil{ZHAW - Zurich}

% Project
\newcommand{\project}{Projektarbeit Aquaponik}
\newcommand{\outline}{Anforderungsanalyse und Umsetzung einer Software-Lösung im Bereich Aquaponik am Beispiel der ZHAW Wädenswil}

% Title
\title{
	\Huge{}\color{blue}\textbf{\project}\\ 
	\vspace{2cm}
	\large{}\color{black}\textbf{\outline}
}

% Define titlepage layout
\makeatletter
\def\@maketitle{%
	\newpage
	\null
	\vskip 1cm%
	\begin{center}%
		\let \footnote \thanks
		{\LARGE \@title \par}%
		\vskip 2cm%
		{\large
			\lineskip .25em%
			\begin{tabular}[t]{c}%
				\@author
			\end{tabular}\par}%
		\vfill%
		{\large \@date}
	\end{center}%
	\par
	\vskip 1.5em
}
\makeatother

%------------------------------------------------------------

\begin{document}
	\sloppy
	\pagenumbering{roman}
	
	\begin{titlepage}
		\maketitle
		\thispagestyle{empty}
	\end{titlepage}
	
	\tableofcontents
	\newpage
	
	% Page numbering is arabic henceforth
	\pagenumbering{arabic}
	\fussy
	

		
	\section{Einleitung}
	
	\subsection{Ausgangslage}
	Ein Aquaponics Projekt besteht bereits aus folgenden Komponenten. Sensoren der Aquaponics Systeme, welche an SC1000 Geräten angeschlossen sind. 
	Ein serieller Bus verknüpft alle SC1000 mit einem RasPi welches über eine Modbus-API die Sensordaten auf eine MySQL Datenbank ablegt. Diese Daten werden auf der Webseite dargestellt.
	

	Die Webseite, welche unter myaquaculturefarm.ch zu finden ist, wird auf hosttech.ch gehostet. Hosttech verwendet als Backend Technologie PHP.
	Unsere Konfigurationsseite wird ebenfalls auf dieser Domain parallel zu den anderen Webseiten von ZHAW Life Sciences und Facility Management gehostet.
	
	Beim Backend sind wir gebunden was die Host-Firma uns zur Verfügung stellt, in diesem Falle wäre das PHP.
	
	\subsection{Aufgabenstellung}
	Die Datenbank, in der die Sensordaten geloggt werden, besteht aus zwei Tabellen. In einer der Tabellen werden die Sensoren eingetragen, die sich in den Systemen befinden und in der zweiten Tabelle werden die Sensordaten abgespeichert und mit dem jeweiligen Sensor verknüpft.
	
	Das Bearbeiten dieser Zuordnungstabelle ist für die Mitarbeiter/Studierende der ZHAW Life Sciences und Facility Management mit dem von Hosttech gegebenen Tool «phpMyAdmin» nicht verständlich. Zusätzlich müssen spezifische Werte eingegeben werden die einen Informatik Laien nicht bekannt sind, welches zu inkorrekte Angabe von Daten führen kann, welches wiederum zu einem Durcheinander in der Log-Tabelle führt. 
	Das «phpMyAdmin» Tool ist ebenfalls nur per Verwaltungsseite der Hosttech Domain erreichbar welches eine zusätzliche Hürde darstellt.
	
	\subsection{Zielsetzung}
	Um die Zuordnuntabelle einfacher zu bearbeiten, soll der Ablauf abgändert werden. Als Lösung stellen wir eine REST Schnittstelle zur Verfügung über welche die Tabelle mit wenigen Handgriffen verändert werden kann. 
	Diese soll übersichtlich und einfach zu bedienen sein. Damit die Schnittstelle zu jeder Zeit erreichbar ist soll sie gehostet werden.
	
	\section{Grundlagen}
	
	\subsection{Hosting}
	Unser Produkt soll auf dem neusten Stand der Dinge sein und daher verzichten wir auf den Hoster Hosttech der schon für das bestehende System verwendet wird. Ein grosser Nachteil von Hosttech ist das wir auf ein PHP Backend beschränkt sind. Um auch das Deployment zu modernisieren und hauptsächlich zu automatisieren möchten wir mit Dockercontainer arbeiten. Hierbei hat unser Betreuer seine Plattform angeboten. Auf dieser können wir ein Image des Dockercontainers hosten lassen. Somit ist dieses zu jederzeit erreichbar. Dies gilt für das Frontend sowie für das Backend.
	Damit der Server nach der Umsetzung nach wie vor erreichbar ist, wurde uns vom Betreuer folgende Seite vorgeschlagen: https://www.hetzner.com/de/
	
	\subsection{Backend Technologie}
	Mit dem Loslösen von Hosttech sind wir frei die Backend Technologie zu bestimmen und wir haben uns auf Java/Spring-Boot geeinigt. Wir haben uns für Spring-Boot entschieden, da alle Entwickler mit Java vertraut sind und OpenAPI Spring-Boot unterstützt. Spring-Boot eignet sich auch gut für unsere Aufgabenstellung, da wir eine Datenbank anschliessen werden und die Daten auf der Webseite nicht nur dargestellt werden sondern auch manipuliert werden können, eignet sich Spring-Boot mit der einfachen Handhabung von JPA gut.
	
	\subsection{Schnittstelle}
	Als Schnittstelle zwischen Backend und Frontend wollen wir das State of the Art Tool OpenAPI einsetzen. Nach genauerer Analyse sehen wir einen grossen Nutzen des Tools, da es uns eine grosse Unterstützung zur Entwicklung der REST Schnittstelle sein wird. In OpenAPI kann in einem YAML File die ganze REST Schnittstelle erstellt und konfiguriert werden daraus kann dann OpenAPI unser Spring-Boot Backend erzeugen was uns viel Arbeit abnimmt.
	
	\subsection{Frontend Technologie}
	Als Frontend-Framework sind wir frei, da es keinen Wunsch/Voraussetzung vom Kunden gab.
	Wir als Gruppe haben uns für Angular(2+) entschieden, da die Mehrheit der Gruppe sich mit diesem Framework gut auskennt. 
	
	\subsection{Datenbank}
	Die bestehende Datenbank auf Hosttech wird in einen neuen Dockercontainer migriert und neben dem Spring-Boot projekt gehostet. Nur das Spring-Boot Projekt hat direkten Zugriff auf die Datenbank, alle Änderungen müssen über die REST Schnittstelle der Spring-Boot Applikation erfolgen. 
	
	\section{Vorgehen / Methoden}
	
	\subsection{Swagger}
	Erste Umsetzungen finden mithilfe eines Swaggerprojekts statt. Hier kann ein Entwurf erstellt werden. Mithilfe dieses Entwurfs wird ein Grundgerüst hergestellt, welcher dazu dient, vereinfacht die REST Schnittstelle zu bauen.

	\begin{figure}[h]
		\centering
		\includegraphics[scale=0.4]{Swagger}
		\caption{Swagger Beispiel}
		\label{fig:Swagger}
	\end{figure}

	Die Schnittstelle soll dazu fähig sein Daten der Sensoren zurückzugeben. Die Möglichkeit soll bestehen, dass die Sensorendaten zusätzlich editierbar sind.
	
	Als Vereinfachung können die Sensoren auch als eine Liste angezeigt werden.
	
	\subsection{Docker}
	Docker kann sehr vielfältig eingesetzt werden, aber worin Docker seine stärken hat ist im Deployment. Mit einem gutem Docker-Compose File ist ein komplexes Multi-Container System mit einem Befehl aufgesetzt.
	
	\subsubsection{Docker-Image}
	Von unserer Spring-Boot Applikation erstellen wir ein Image welches auf DockerHub hochgeladen wird. 
	
	\subsubsection{Docker-Compose}
	Das Docker-Compose holt sich die neuste Version der Applikation und erstellt ein Netzwerk für das Gesamte PAAQ-System. Der Applikation wird ein Portforwarding eingerichtet, dass diese auch von Aussen zugänglich ist. Anschliessend wird auch ein MySQL-Container heruntergeladen und in das gleiche Netzwerk eingehängt, dass eine Kommunikation zwischen Spring-Boot Applikation und MySQL-Datenbank statt finden kann. Der MySQL Datenbank wird noch der Dump der alten Datenbank mitgegeben, dass eine exakte Kopie der vorherigen Datenbank erstellt.
	
	\section{Resultate}
	\section{Diskussion und Ausblick}
	
	\subsection{Sprint 0 Meeting}
	\begin{itemize}
		\item Angular Demo von  Herr Bachmann TodoAngular.
		\begin{itemize}
			\item In Github Frontend - Package.JSON ist alles vorhanden was zu diesem Thema als Wissen benötigt wird.
		\end{itemize}
		\item Hosting kann von Herrn Bachmann übernommen werden.
		\begin{itemize}
			\item Hierfür muss entweder ein Dockercontainer für das Frontend und Backend respektive erstellt werden, oder für das Frontend und Backend kombiniert.
		\end{itemize}
		\item Cloud Service für VM Möglichkeiten.
		\begin{itemize}
			\item Auf der Homepage \url{https://www.hetzner.com} gibt es zahlreiche kosteneffiziente und interessante Angebote für verschiedene Services.
		\end{itemize}
		\item Erstellen und diskutieren der Tasks für Sprint 1.
	\end{itemize}	
	
	\subsection{Sprint 1 Meeting}
	\begin{itemize}
		\item Durch verschiedene Versionierungen und unterschiedlichen Arbeitsstationen kam es zu einigen Problemen mit Gradle unter anderem auch Dependeciesfehler.
		\item Unter längerer Diskussion wurde beschlossen dass das Frontend und Backend in einem Dockercontainer kombiniert werden um das hosten und arbeiten zu vereinfachen.
		\item Eine kurze Dockerpräsentation auf einer virtuellen Ubuntumaschine über den aktuellen Stand wird vorgeführt.
		\item Angular Frontend steht noch offen und wird als Letztes angegangen.
		\item Ein weiterer Punkt wäre ein mögliches Treffen mit dem Team aus Wädischwil bezüglich Mockups. Um eine grobe Richtung für ungefähre GUI Vorstellungen zu bekommen oder ob dies komplett nach eigenem Ermessen erstellt werden soll.
	\end{itemize}	
	\subsection{Sprint 2 Meeting}
	\begin{itemize}
		\item Um weiter frotzufahren mit dem GUI
		\item Erklärung Docker Benutzung
		\item Andere Hostverwendung (Hetzner)
		\item Tabellarisch für Einstellungen der Sensoren
		\item Reihenfolge gleich wie bei SC1000 geregelt
		\item Adressen der Sensoren werden im SC1000 geregelt
		\item Daten werden über MQTT zur Urbanblue geschickt (PHP)
		\item Zeitfenster 2 Wochen bis Frontend fertig	
	\end{itemize}
	\section{Verzeichnisse}
	\subsection{Literaturverzeichnis}
	\subsection{Glossar}
	\subsection{Abbildungsverzeichnis}
	\section{Anhang}
		
		
		
\end{document}