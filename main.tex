\documentclass[a4paper]{article}

%------------------------------------------------------------
% PACKAGES
% Base Packages
\usepackage[ngerman]{babel}
\usepackage[utf8]{inputenc}
\usepackage[T1]{fontenc}
\usepackage{geometry}
\usepackage{subfiles}
\usepackage{xcolor}
\usepackage{graphicx}
\usepackage{pdfpages}
\usepackage{todonotes}
\usepackage{listings}
\usepackage[backend=bibtex, style=ieee, citestyle=ieee]{biblatex}
\usepackage{csquotes}

% Misc Packages
\usepackage{svg}
\usepackage{fancyhdr}
\usepackage{authblk}

% End Packages
%------------------------------------------------------------

%------------------------------------------------------------
% DEFINITIONS
% Color Definitions
\definecolor{codegreen}{rgb}{0,0.6,0}
\definecolor{codegray}{rgb}{0.5,0.5,0.5}
\definecolor{codepurple}{rgb}{0.58,0,0.82}
\definecolor{backcolour}{rgb}{0.921, 0.929, 0.937}

% Listings Definitions
\lstdefinestyle{mystyle}{
	backgroundcolor=\color{backcolour},   
	commentstyle=\color{codegreen},
	keywordstyle=\color{magenta},
	numberstyle=\tiny\color{codegray},
	stringstyle=\color{codepurple},
	basicstyle=\ttfamily\footnotesize,
	breakatwhitespace=false,         
	breaklines=true,                 
	captionpos=b,                    
	keepspaces=true,                 
	numbers=left,                    
	numbersep=5pt,                  
	showspaces=false,                
	showstringspaces=false,
	showtabs=false,                  
	tabsize=2
}
\lstset{style=mystyle}
%------------------------------------------------------------


%------------------------------------------------------------
% SETTINGS
% Bibliography settings
%\addbibresource{example.bib} % Add your own bibtex file here

% Geometry / Margin Settings
\geometry{
	paper=a4paper, % Change to letterpaper for US letter
	inner=2.5cm, % Inner margin
	outer=3.8cm, % Outer margin
	bindingoffset=.5cm, % Binding offset
	top=1.5cm, % Top margin
	bottom=1.5cm, % Bottom margin
	%showframe, % Uncomment to show how the type block is set on the page
}

% Paths
\graphicspath{{images/}}
\svgpath{{images/svg/}}
\pagestyle{fancy}
\fancyhf{}

% End Geometry / Margin Settings
%------------------------------------------------------------

%------------------------------------------------------------
% DOCUMENT SPECIFIC
% Author
\author{
	Deniz Akca
	\and
	Dennis Bannerman
	\and
	Mike Iten
}
\affil{ZHAW - Zurich}

% Project
\newcommand{\project}{Projektarbeit Aquaponik}
\newcommand{\outline}{Anforderungsanalyse und Umsetzung einer Software-Lösung im Bereich Aquaponik am Beispiel der ZHAW Wädenswil}

% Title
\title{
	\Huge{}\color{blue}\textbf{\project}\\ 
	\vspace{2cm}
	\large{}\color{black}\textbf{\outline}
}

% Define titlepage layout
\makeatletter
\def\@maketitle{%
	\newpage
	\null
	\vskip 1cm%
	\begin{center}%
		\let \footnote \thanks
		{\LARGE \@title \par}%
		\vskip 2cm%
		{\large
			\lineskip .25em%
			\begin{tabular}[t]{c}%
				\@author
			\end{tabular}\par}%
		\vfill%
		{\large \@date}
	\end{center}%
	\par
	\vskip 1.5em
}
\makeatother

%------------------------------------------------------------

\begin{document}
	\sloppy
	\pagenumbering{roman}
	
	\begin{titlepage}
		\maketitle
		\thispagestyle{empty}
	\end{titlepage}
	
	\tableofcontents
	\newpage
	
	% Page numbering is arabic henceforth
	\pagenumbering{arabic}
	\fussy
	
	\section{Recherche}
	
	\subsection{Sensordaten}
	Die Sensoren der Systeme sind an SC1000 Geräten angeschlossen. Ein serieller Bus verknüpft alle SC1000 mit einem RasPi welches über eine Modbus-API die Sensordaten auf eine MySQL Datenbank ablegt. Die Daten werden auf der Webseite dargestellt.
	
	\subsection{Host}
	Die myaquaculturefarm.ch Webseite wird auf hosttech.ch gehostet. Hosttech verwendet als Backend PHP.
	Unsere Konfigurationsseite wird ebenfalls auf dieser Domain parallel zu den anderen Webseiten von ZHAW Life Sciences und Facility Management gehostet.
	
	\subsection{Backend Technologie}
	Beim Backend sind wir gebunden was die Host-Firma uns zur Verfügung stellt, in diesem Falle wäre das PHP.
	
	\section{Analyse}
	
	\subsection{Problem}
	Die Datenbank, in der die Sensordaten geloggt werden, besteht aus zwei Tabellen. In einer der Tabellen werden die Sensoren eingetragen, die sich in den Systemen befinden und in der zweiten Tabelle werden die Sensordaten abgespeichert und mit dem jeweiligen Sensor verknüpft.
	
	Das Bearbeiten dieser Zuordnungstabelle ist für die Mitarbeiter/Studierende der ZHAW Life Sciences und Facility Management mit dem von Hosttecht gegebenen Tool «phpMyAdmin» nicht verständlich. Zusätzlich müssen spezifische Werte eingegeben werden die einen Informatik Laien nicht bekannt sind, welches zu inkorrekte Angabe von Daten führen kann, welches wiederum zu einem Durcheinander in der Log-Tabelle führt. Das «phpMyAdmin» Tool ist ebenfalls nur per Verwaltungsseite der Hosttech Domain erreichbar welches eine zusätzliche Hürde darstellt.
	
	\subsection{Lösung}
	Eine einfachere Variante zur Bearbeitung dieser Zuordnungstabelle erschaffen.
	
	\section{Design}
	
	\subsection{Hosting}
	Wir sind nicht zufrieden mit Hosttech und möchten gerne ein anderes Backend als PHP einsetzen, daher haben wir uns entschieden die Verwaltungswebseite auf einem anderen Anbieter zu hosten.
	
	\subsection{Backend Technologie}
	Mit dem loslösen von Hosttech sind wir frei die Backend Technologie zu bestimmen und wir haben uns auf Java/Spring geeinigt. Wir haben uns für Spring entschieden, da alle Entwickler mit Java vertraut sind und OpenAPI Spring unterstützt.
	
	\subsection{OpenAPI}
	Als Schnittstelle zwischen Backend und Frontend wollen wir das State of the Art Tool OpenAPI einsetzen. Nach genaurere Analyse sehen wir einen grossen Nutzen des Tools, da es uns eine grosse Unterstützung zur Entwicklung der REST Schnittstelle sein wird. 
	
	\subsection{Frontend Technologie}
	Als Frontend-Framework sind wir frei, da es keinen Wunsch/Voraussetzung vom Kunden gab.
	Wir als Gruppe haben uns für Angular(2+) entschieden, da die Mehrheit der Gruppe sich mit diesem Framework gut auskennt. 
	
	\section{Umsetzung}
	
	\section{Test}
	
	\section{Fazit}
		
\end{document}