% dual entry command stolen from
% https://en.wikibooks.org/wiki/LaTeX/Glossary#Dual_entries_with_reference_to_a_glossary_entry_from_an_acronym
%
% used like:
% \newdualentry{label} % label
% {LBL}                % abbreviation
% {Label}              % long form
% {What's a label???}  % description
%
% Refer to acronym with \gls{OWD} and the glossary with \gls{gls-OWD}


\usepackage{xparse}

\DeclareDocumentCommand{\newdualentry}{ O{} O{} m m m m } {
	\newglossaryentry{gls-#3}{name={#5},text={#5\glsadd{#3}},
		description={#6},#1
	}
	\makeglossaries
	\newacronym[see={[Glossary:]{gls-#3}},#2]{#3}{#4}{#5\glsadd{gls-#3}}
}




% Acronym

\newacronym{acronymexample}
{AE}
{\textbf{A}cronym \textbf{E}xample - additional description}



% ------------------------------------------------

% Acronyms/Glossary mixed

\newdualentry{mvp}
{MVP}
{\textbf{M}inimal \textbf{V}iable \textbf{P}roduct}
{
	Ein Minimum Viable Product ist das bezogen auf den Umfang kleinstmögliche, auslieferbare und vor allem funktionsfähige Produkt.
}

% ------------------------------------------------

% Glossary 



\newglossaryentry{peerreview}
{
	name=Peer-Review,
	description={
		Peer-Review ist ein Verfahren zur Qualitätssicherung, bei dem ein Erzeugnis durch unabhängige bzw. unbeteiligte Gutachter auf seine ausreichende Qualität überprüft wird.
	}
}



\newglossaryentry{jira}
{
	name=JIRA,
	description={
		JIRA ist ein Produkt der Firma Atlassian, das Teams in unterschiedlichen Bereichen des Projektmanagements unterstützen soll.
	}
}


\newglossaryentry{cicd}
{
	name=CI/CD,
	description={
		CI/CD (Continuous Integration / Continuous Delivery) beschreibt eine Praktik in der Software-Entwicklung, die dafür sorgt, dass die zu entwickelnde Software konstant in ein ausführbares Programm umgewandelt und somit von allen Stakeholdern getestet werden kann.
	}
}

\newglossaryentry{versionierung}
{
	name=Versionierung,
	description={
		Mittels Versionierung wird in der Software-Entwicklung eine reibungslose Kooperation der Entwickler ermöglicht. Ausserdem kann so eine Chronologie aller gemachten Veränderungen eingesehen werden.
	}
}

% ------------------------------------------------
