% dual entry command stolen from
% https://en.wikibooks.org/wiki/LaTeX/Glossary#Dual_entries_with_reference_to_a_glossary_entry_from_an_acronym
%
% used like:
% \newdualentry{label} % label
% {LBL}                % abbreviation
% {Label}              % long form
% {What's a label???}  % description
%
% Refer to acronym with \gls{OWD} and the glossary with \gls{gls-OWD}


\usepackage{xparse}

\DeclareDocumentCommand{\newdualentry}{ O{} O{} m m m m } {
	\newglossaryentry{gls-#3}{name={#5},text={#5\glsadd{#3}},
		description={#6},#1
	}
	\makeglossaries
	\newacronym[see={[Glossary:]{gls-#3}},#2]{#3}{#4}{#5\glsadd{gls-#3}}
}




% Acronym

\newacronym{acronymexample}
{AE}
{\textbf{A}cronym \textbf{E}xample - additional description}



% ------------------------------------------------

% Acronyms/Glossary mixed

\newdualentry{cicd}
	{CI/CD}
	{\textbf{C}ontinuous \textbf{I}ntegration / \textbf{C}ontinuous \textbf{D}elivery}
	{
			CI/CD (Continuous Integration / Continuous Delivery) beschreibt eine Praktik in der Software-Entwicklung, die dafür sorgt, dass die zu entwickelnde Software konstant in ein ausführbares Programm umgewandelt und somit von allen Stakeholdern getestet werden kann.\cite{CI/CD}
	}

\newdualentry{php}
	{PHP}
	{\textbf{P}ersonal Home Page \textbf{H}ypertext \textbf{P}reprocessor}
	{
		Eine Skript-Sprache die für die Verwendung in der Webentwicklung entwickelt wurde. Wird meistens als Web-Server verwendet.\cite{PHP}
	}

\newdualentry{api}
	{API}
	{\textbf{A}pplication \textbf{P}rogramming \textbf{I}nterface}
	{
		Eine Programmschnittstelle welche anderen Programmen eine Anbindung bietet.\cite{API}
	}

\newdualentry{rest}
	{REST}
	{\textbf{R}epresentational \textbf{S}tate \textbf{T}ransfer}
	{
		REST ist eine Softwarearchitektursstil welche über HTTP und HTTPS kommuniziert. Mit REST können API einheitlich und einfach angesteuert werden um Daten zwischen zwei Systemen auszutauschen.\cite{REST}
	}

\newdualentry{jpa}
	{JPA}
	{\textbf{J}ava \textbf{P}ersistence \textbf{\gls{api}}}
	{
		Java Persistence API ist eine Schnittstelle in Java für die Zuordnung und Übertragung von Objekten zu Datenbankeinträgen vereinfacht.\cite{JPA}
	}

\newdualentry{paaq}
	{PAAQ}
	{\textbf{P}rojektarbeit \textbf{A}quaponik}
	{
		Der Projektname den wir unserer Projektarbeit gegeben haben.
	}

% ------------------------------------------------

% Glossary 

\newglossaryentry{dump}
{
	name=Dump,
	description={
		Eine in SQL formatierte kopie einer Datenbank.
	}
}


\newglossaryentry{peerreview}
{
	name=Peer-Review,
	description={
		Peer-Review ist ein Verfahren zur Qualitätssicherung, bei dem ein Erzeugnis durch unabhängige bzw. unbeteiligte Gutachter auf seine ausreichende Qualität überprüft wird.\cite{Peer-Review}
	}
}

\newglossaryentry{mysql}
{
	name=MySQL,
	description={
		Eine weit verbreitete Open-Source Datenbank von Oracle.
	}
}

\newglossaryentry{jira}
{
	name=JIRA,
	description={
		JIRA ist ein Produkt der Firma Atlassian, das Teams in unterschiedlichen Bereichen des Projektmanagements unterstützen soll.\cite{JIRA}
	}
}

\newglossaryentry{versionierung}
{
	name=Versionierung,
	description={
		Mittels Versionierung wird in der Software-Entwicklung eine reibungslose Kooperation der Entwickler ermöglicht. Ausserdem kann so eine Chronologie aller gemachten Veränderungen eingesehen werden.\cite{Versionierung}
	}
}

\newglossaryentry{sc1000}
{
	name=SC1000,
	description={
		Der Universalcontroller SC1000 von Hach ist ein modulares Steuerungssystem mit Anschlüssen bis zu 8 Sensoren. Dieser Controller kann mit weiteren Controllern über einen Bus kommunizieren.\cite{SC1000}
	}
}

\newglossaryentry{raspberrypi}
{
	name=Raspberry Pi,
	description={
		Ein Einplatinencomputer mit einem System on a Chip Design ausgestattet mit einer ARM-CPU. Auf den GPIO Pins können weitere Hardware Module angeschlossen werden.
	}
}

\newglossaryentry{modbusapi}
{
	name=Modbus-API,
	description={
		Eine API für um die Sensordaten abzufragen/zusenden, welche über den seriellen Modbus abgefragt werden.\cite{Modbus-API}
	}
}

\newglossaryentry{hosttech}
{
	name=Hosttech,
	description={
		Eine schweizer Webhosting-Firma.
	}
}

\newglossaryentry{sensor}
{
	name=Sensor/en,
	description={
		In Gebrauch sind verschiedene Sensortypen die unterschiedliche Werte auswerten können wie Sauerstoffgehalt, Temperatur und PH Werte. Einige dieser Sensoren sind auch in der Lage mehrere verschiedene Werte auszulesen.
	}
}

\newglossaryentry{zuordnungstabelle}
{
	name=Zuordnungstabelle,
	description={
		In der Zuordnungstabelle sind alle Sensoren und deren Identifizierung wie auch Sensortyp vorhanden. Zusätzlich werden Adressdaten gespeichert, damit das \gls{raspberrypi} den Sensor zum richtigen Sensorwert verknüpfen kann.
	}
}

\newglossaryentry{phpmyadmin}
{
	name=phpMyAdmin,
	description={
		Basierend auf \gls{php} ist das ein benutzerfreundliches Datenbank Bedinungstool, welche auch \gls{mysql} Datenbanken verwalten kann.
	}
}

\newglossaryentry{logtabelle}
{
	name=Logtabelle,
	description={
		Neben der \gls{zuordnungstabelle} gibt es eine Logtabelle die die Sensordaten speichert.
	}
}

\newglossaryentry{domain}
{
	name=Domain,
	description={
		Eine Domain ist ein eindeutiger Name einer Internetadresse. Die Domains werden von einem Domain Name System in eine Internetadresse umgewandelt. Eine Domain ist für einen Menschen einfacher zu merken als eine Zahlenfolge.
	}
}

\newglossaryentry{deployment}
{
	name=Deployment,
	description={
		Prozess von Verteilung und Installation von Software auf einem oder mehreren Rechner.
	}
}

\newglossaryentry{docker}
{
	name=Docker,
	description={
		Eine Containervirtualisierung und Verwaltungssoftware.\cite{Docker}
	}
}

\newglossaryentry{container}
{
	name=Container,
	description={
		Eine virtualisierte Instanz des Kernels des Hostsystems, im Gegensatz zu zur Virtualisierung mittels Hypervisors braucht weniger Ressourcen ist jedoch auf das Hostsystem limitiert.
	}
}

\newglossaryentry{image}
{
	name=Image,
	description={
		Ein Abbild eines Containers.
	}
}

\newglossaryentry{springboot}
{
	name=Spring Boot,
	description={
		Java Web Framework für die einfache Entwicklung von Java EE.
	}
}

\newglossaryentry{openapi}
{
	name=OpenAPI,
	description={
		Open API ist ein Standard zur Beschreibung von \gls{rest} \gls{api}.\cite{OpenAPI}
	}
}

\newglossaryentry{yaml}
{
	name=YAML,
	description={
		YAML ist eine verinfachte Auszeichnungssprache die angelehnt ist an XML.\cite{YAML}
	}
}

\newglossaryentry{github}
{
	name=GitHub,
	description={
		GitHub dient als netzbasierte Versionsverwaltung für Software-Projekte. GitHub basiert sich auf \gls{git}.
	}
}

\newglossaryentry{git}
{
	name=Git,
	description={
		Git ist ein Versionsverwaltungstool für Dateien.
	}
}

\newglossaryentry{dockerhub}
{
	name=DockerHub,
	description={
		Auf DockerHub befinden sich \gls{repository} zur Versionsvewaltung von \gls{docker} \gls{container}.
	}
}

\newglossaryentry{repository}
{
	name=Repository,
	description={
		Ein verwaltetes Verzeichnis zur Speicherung von Dateien.
	}
}

\newglossaryentry{mainbranch}
{
	name=Main Branch,
	description={
		Der Haupt \gls{branch} in dem sich die aktuellste laufende Version befindet.
	}
}

\newglossaryentry{branch}
{
	name=Branch,
	description={
		Ein Branch ist eine Abspaltung in der Softwareversion. Diese werden in \gls{git} verwendet um Features zu implementieren ohne den \gls{mainbranch} zu beeinflussen.
	}
}

\newglossaryentry{commit}
{
	name=Commit,
	description={
		Ein Commit ist eine bestätigte Freischaltung einer oder mehrere Änderungen am Code.
	}
}

\newglossaryentry{pullrequest}
{
	name=Pull Request,
	description={
		Ein Pull Request ist eine Anfrage um eine Änderung oder einen Branch in einen weiteren Branch \gls{merge} zu lassen. Wird ein Pull Request genehmigt wird ein \gls{commit} ausgeführt.
	}
}

\newglossaryentry{merge}
{
	name=Merge,
	description={
		Wenn ein Feature in einem Feature \gls{branch} in ein \gls{mainbranch} hinzugefügt werden soll, muss ein Merge ausgeführt werden, die die zwei \gls{branch} zusammenfügt.
	}
}

\newglossaryentry{push}
{
	name=Push,
	description={
		Eine \gls{commit} auf \gls{github} hochladen.
	}
}

\newglossaryentry{githubactions}
{
	name=Github Actions,
	description={
		Eine \gls{cicd} implementierung von \gls{github}.
	}
}

\newglossaryentry{dockercompose}
{
	name=Docker Compose,
	description={
		Wird benutzt um ein Docker-Compose \gls{yaml} File ausführen zu können. Docker-Compose steuert dann \gls{docker} an.
	}
}

\newglossaryentry{portforwarding}
{
	name=Portforwarding,
	description={
		Ein Netzwerkrouter blockt normalerweise die meisten Ports von Aussen. Falls Dienste des Netzes im Internet verfügbar werden sollen, kann mittels Portforwarding ein Port vom Netzwerkrouter mit einem Rechner aus dem internen Netz verknüpft werden. Somit leitet der Netwerkrouter alle Pakete an den richtigen Dienst.
	}
}

\newglossaryentry{angular}
{
	name=Angular,
	description={
		Angular ist ein TypeScript basiertes Front-End-Webapplikationsframework.\cite{Angular}
	}
}

\newglossaryentry{githubfork}
{
	name=GitHub-Fork,
	description={
		Eine Art \gls{branch} auf \gls{repository} Ebene.\cite{Github-Fork}
	}
}

\newglossaryentry{sloppyio}
{
	name=Sloppy.io,
	description={
		Sloppy.io ist ein \gls{docker} Hostingdienst.
	}
}

\newglossaryentry{boilerplate}
{
	name=Boilerplate-Code,
	description={
		Ein gängiger Begriff für Code, der keine signifikante Logik beinhaltet, generell simpel ist oder aus Programmiersprachen bedingten Begebenheiten sehr oft wiedehrolt werden muss.
	}
}

\newglossaryentry{interfacefirst}
{
	name=Interface-First,
	description={
		Bezeichnet die Methode, zuerst die Schnittstelle mit ihren Aktionen und jeweiligen Datenfeldern zu definieren, bevor die Entwicklung des Back- und Frontends beginnt.
	}
}
% ------------------------------------------------
